\documentclass[
notumble,
nofoldmark,
]{leaflet}

\newcommand{\lsSeries}{tmnlp}
\usepackage{lspflyer}

\begin{document} 
\lspflyer{    \footnotesize

The book series centres around human and machine translation,
with a special emphasis on empirical studies. This includes
computational, corpus linguistic and cognitive aspects of
translation. By its nature, the topic of translation is
interdisciplinary. In an effort to narrow the gap between
disciplines, the series is not limited to translation
studies, but is also interested in related fields such as
computational linguistics, language technology, psycholinguistics,
neurolinguistics, cognitive science and corpus linguistics --
provided that there is a link to human, computer-assisted, or
machine translation, or to any type of multilingual natural
language processing.


  }
{
    \section{Possible Topics} 

\raggedright 
\begin{itemize}
 \item [$\rangle$]machine translation (statistical, rule-based, example-based and hybrid), machine aided translation and interpreting
translation technology\item [$\rangle$]
use of corpora in translation\item [$\rangle$]
annotation, alignment and searchability of translation data\item [$\rangle$]
cognitive aspects of translation\item [$\rangle$]
modelling the translation process\item [$\rangle$]
multi-modal and audiovisual translation 
\end{itemize}


    \section{Editorial Board}    

    \begin{itemize}
    \item[$\rangle$] Reinhard Rapp (Chief Editor, Aix-Marseille Universit\'e)
    \item[$\rangle$] Silvia Hansen-Schirra (Johannes Gutenberg- Universit\"at Mainz)
    \item[$\rangle$] Oliver \v{C}ulo (Johannes Gutenberg-Universit\"at Mainz)
    \end{itemize} 
}{LIGHTGRAY}
\end{document} 
