\documentclass[
notumble,
nofoldmark,
]{leaflet}

\newcommand{\lsSeries}{tmnlp}
\usepackage{lspflyer}

\begin{document} 
\lspflyer{    \small
The book series centres around human and machine translation, with a special emphasis on empirical studies. This includes computational, corpus linguistic and cognitive aspects of translation.
By its nature, the topic of translation is interdisciplinary in the sense that it involves many of the classical linguistic sub-disciplines such as computational linguistics, corpus linguistics, morphology, syntax, semantics, pragmatics, text linguistics, lexicography, psycholinguistics, neurolinguistics, applied linguistics and others. 
% However, all book submissions need to have a clear focus on the translation aspect, and a special emphasis is laid on empirical studies. The aim of the book series is to bring these different perspectives closer together by offering a forum for all different approaches to the empirical study of translation. The series welcomes in particular studies investigating corpus data and/or experimental findings, preferably in – but not limited to -- a quantitative perspective. 
  } 
{
    \color{LIGHTGRAY}
    \section{Possible Topics} 
 
\raggedright
\begin{itemize}
 \item [$\rangle$]machine translation (statistical, rule-based, example-based and hybrid), machine aided translation and interpreting
translation technology\item [$\rangle$]
use of corpora in translation\item [$\rangle$]
annotation, alignment and searchability of translation data\item [$\rangle$]
cognitive aspects of translation\item [$\rangle$]
modelling the translation process\item [$\rangle$]
multi-modal and audiovisual translation 
\end{itemize}



    \section{Editorial Board}    

    \begin{itemize}
    \item[$\rangle$] Reinhard Rapp (Chief Editor, Aix-Marseille Universit\'e)
    \item[$\rangle$] Silvia Hansen-Schirra (Johannes Gutenberg-Universit\"at Mainz)
    \item[$\rangle$] Oliver \v{C}ulo (Johannes Gutenberg-Universit\"at Mainz)
    \end{itemize} 
}
\end{document} 
