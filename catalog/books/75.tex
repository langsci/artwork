\btitle{Linguistic variation, identity construction and cognition}
\bsubtitle{}
\bauthor{Katie Drager}
\bisbn{3415215213}
\bdoi{153256261}  
\bprice{\priceM}
\bpages{200}
\bappeared{forthcoming}
\blanguage{English}
\btagline{Speakers use a variety of different linguistic resources in the construction of their identities, and do so because their mental representations of linguistic and social information are linked. }
\bblurb{Speakers use a variety of different linguistic resources in the construction of their identities, and they are able to do so because their mental representations of linguistic and social information are linked.
While the exact nature of these representations remains unclear, there is growing evidence that they encode a great deal more phonetic detail than traditionally assumed and that the phonetic detail is linked with word-based information. This book investigates the ways in which a lemma's phonetic realisation depends on a combination of its grammatical function and the speaker's social group. This question is investigated within the context of the word like as it is produced and perceived by students at an all girls' high school in New Zealand. }