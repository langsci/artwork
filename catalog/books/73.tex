\btitle{Grammaticalization in the North: Noun phrase morphosyntax in Scandinavian vernaculars}
\bauthor{Östen Dahl}
\bisbn{5235325325}
\bdoi{42135235235} 
\bprice{26 EUR}
\btagline{This book looks at some phenomena within the grammar of the noun phrase in a group of traditional North Germanic varieties mainly spoken in Sweden and Finland. }
\bblurb{This book looks at some phenomena within the grammar of the noun phrase in a group of traditional North Germanic varieties mainly spoken in Sweden and Finland, usually seen as Swedish dialects, although the differences between them and Standard Swedish are often larger than between the latter and the other standard Mainland Scandinavian languages.
In addition to being conservative in many respects – e.g. in preserving nominal cases and subject-verb agreement – these varieties also display many innovative features.}