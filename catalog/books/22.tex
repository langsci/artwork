\btitle{The Alor-Pantar languages}
\bsubtitle{History and typology}
\bauthor{Marian Klamer}
\bisbnsc{978-3-944675-61-9}
\bisbnhc{978-3-944675-60-2}
\bprice{30}{60}
\bpages{479}
\bappeared{2014}
\blanguage{English}
\btagline{The Alor-Pantar family constitutes the westernmost outlier group of Papuan (Non-Austronesian) languages.}
\bblurb{The Alor-Pantar family constitutes the westernmost outlier group of Papuan (Non-Austronesian) languages.
Its twenty or so languages are spoken on the islands of Alor and Pantar, located just north of Timor, in eastern Indonesia. Together with the Papuan languages of Timor, they make up the Timor-Alor-Pantar family. The languages average 5,000 speakers and are under pressure from the local Malay variety as well as the national language, Indonesian.
 }