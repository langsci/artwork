\btitle{ Syntax und Valenz}
\bsubtitle{Zur Modellierung kohärenter und elliptischer Strukturen mit Baumadjunktionsgrammatiken}
\bauthor{Timm Lichte}
\bisbnsc{978-3-944675-63-3}
\bisbnhc{978-3-944675-62-6}
\bprice{30}{60}
\bpages{420}
\bappeared{2015}
\blanguage{German}
\btagline{Diese Arbeit untersucht das Verhältnis zwischen Syntaxmodell und lexikalischen Valenzeigenschaften anhand  der Phänomenbereiche Kohärenz und Ellipse. }
\bblurb{Diese Arbeit untersucht das Verhältnis zwischen Syntaxmodell und lexikalischen Valenzeigenschaften anhand von Baumadjunktionsgrammatiken (TAG) und anhand der Phänomenbereiche Kohärenz und Ellipse.
Wie die meisten prominenten Syntaxmodelle betreibt TAG eine Amalgamierung von Syntax und Valenz, die oft zu Realisierungsidealisierungen führt. Es wird jedoch gezeigt,
a) dass TAG dabei gewisse Realisierungsidealisierungen vermeidet und Diskontinuität bei Kohärenz direkt repräsentieren kann;
    b) dass TAG trotzdem und trotz der   eingeschränkten Ausdrucksstärke zu einer linguistisch sinnvollen Analyse kohärenter Konstruktionen herangezogen werden kann;
    c)  dass der TAG-Ableitungsbaum für die indirekte Gapping-Modellierung eine ausreichend informative Bezugsgröße darstellt. }