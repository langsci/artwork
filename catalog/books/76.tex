\btitle{New directions in corpus-based translation studies}
\bsubtitle{}
\bauthor{Claudio Fantinuoli, Federico Zanettin }
\bisbn{234q235235}
\bdoi{2352352}  
\bprice{\priceM}
\bpages{200}
\bappeared{2015}
\blanguage{English}
\btagline{Corpus-based translation studies has become a major paradigm and research methodology and has investigated a wide variety of topics in the last two decades. }
\bblurb{Corpus-based translation studies has become a major paradigm and research methodology and has investigated a wide variety of topics in the last two decades.
The contributions to this volume add to the range of corpus-based studies by providing examples of some less explored applications of corpus analysis methods to translation research. They show that the area keeps evolving as it constantly opens up to different frameworks and approaches, from appraisal theory to process-oriented analysis, and encompasses multiple translation settings, including (indirect) literary translation, machine(-assisted) translation and the practical work of professional legal translators. The studies included in the volume also expand the range of application of corpus applications in terms of the tools used to accomplish the research tasks outlined.  }