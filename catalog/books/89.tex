\btitle{The empirical base of linguistics}
\bsubtitle{Grammaticality judgments and linguistic methodology}
\bauthor{Carson T. Schütze}
\bisbnsc{978-3-946234-04-3}
\bisbnhc{978-3-946234-03-6}
\bprice{30}{50}  
\bpages{266}
\bappeared{2016}
\blanguage{English}
\btagline{Carson T. Schütze presents here a detailed critical overview of the vast literature on the nature and utility of grammaticality judgments and other linguistic intuition. }
\bblurb{Carson T. Schütze presents here a detailed critical overview of the vast literature on the nature and utility of grammaticality judgments and other linguistic intuitions, and the ways they have been used in linguistic research. He shows how variation in the judgment process can arise from factors such as biological, cognitive, and social differences among subjects, the particular elicitation method used, and extraneous features of the materials being judged. He then assesses the status of judgments as reliable indicators of a speaker's grammar.  }