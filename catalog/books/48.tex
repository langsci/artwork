\btitle{Natural causes of language}
\bsubtitle{Frames, biases, and cultural transmission}
\bauthor{N.J. Enfield}
\bisbn{124215425235}
\bdoi{3466643}  
\bprice{\priceXS}
\bpages{100}
\bappeared{2014}
\blanguage{English}
\btagline{What causes a language to be the way it is? Some features are universal, some are inherited, others are borrowed, and yet others are internally innovated. }
\bblurb{What causes a language to be the way it is? Some features are universal, some are inherited, others are borrowed, and yet others are internally innovated.
But no matter where a bit of language is from, it will only exist if it has been diffused and kept in circulation through social interaction in the history of a community. This book makes the case that a proper understanding of the ontology of language systems has to be grounded in the causal mechanisms by which linguistic items are socially transmitted, in communicative contexts. A biased transmission model provides a basis for understanding why certain things and not others are likely to develop, spread, and stick in languages. Because bits of language are always parts of systems, we also need to show how it is that items of knowledge and behavior become structured wholes.  }