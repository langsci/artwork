\btitle{A grammar of Pichi}
\bauthor{Kofi Yakpo}
\bisbn{145123523112}
\bdoi{53215236236} 
\bprice{41 EUR} 
\btagline{Pichi is an Afro-Caribbean English Lexifier Creole spoken by some 150,000 people on the island of Bioko, Equatorial Guinea. }
\bblurb{Pichi is an Afro-Caribbean English Lexifier Creole spoken by some 150,000 people on the island of Bioko, Equatorial Guinea.
The language is an off-shoot of Krio (Sierra Leone) and shares many characteristics with closely related languages in West Africa and the Caribbean, such as Nigerian Pidgin and Sranan (Suriname). However, the isolation of Pichi from English and Krio, extensive contact and hybridization with Spanish, language shift from the Bantu language Bubi, as well as koineization through the prolonged coexistence with Nigerian and Cameroon Pidgin have given the language a distinct character. }