\btitle{Grammatical theory}
\bsubtitle{From transformational grammar to constraint-based approaches}
\bauthor{Stefan Müller}
\bisbn{234q3525}
\bdoi{5235325}  
\bprice{\priceXL}
\bpages{800}
\bappeared{forthcoming}
\blanguage{English}
\btagline{This book introduces formal grammar theories that play a role in current linguistics or contributed tools that are relevant for current linguistic theorizing}
\bblurb{This book introduces formal grammar theories that play a role in current linguistics or contributed tools that are relevant for current linguistic theorizing (Phrase Structure Grammar, Transformational Grammar/Government \& Binding, Mimimalism, Generalized Phrase Structure Grammar, Lexical Functional Grammar, Categorial Grammar, Head-Driven Phrase Structure Grammar, Construction Grammar, Tree Adjoining Grammar, Dependency Grammar).
The key assumptions are explained and it is shown how each theory treats arguments and adjuncts, the active/passive alternation, local reorderings, verb placement, and fronting of constituents over long distances. The analyses are explained with German as the object language.
% In a final part of the book the approaches are compared with respect to their predictions regarding language acquisition and psycholinguistic plausibility. The nativism hypothesis that claims that humans posses genetically determined innate language-specific knowledge is examined critically and alternative models of language acquisition are discussed. In addition this more general part addresses issues that are discussed controversially in current theory building such as the question whether flat or binary branching structures are more appropriate, the question whether constructions should be treated on the phrasal or the lexical level, and the question whether abstract, non-visible entities should play a role in syntactic analyses. It is shown that the analyses that are suggested in the various frameworks are often translatable into each other. The book closes with a section that shows how properties that are common to all languages or to certain language classes can be captured.
%
% The book is a translation of the German book Grammatiktheorie, which was published by Stauffenburg in 2010. The following quotes are taken from reviews:
%
% With this critical yet fair reflection on various grammatical theories, Müller fills what was a major gap in the literature. Karen Lehmann, Zeitschrift für Rezen­sio­nen zur ger­man­is­tis­chen Sprach­wis­senschaft, 2012
%
% Stefan Müller’s recent introductory textbook, Gram­matik­the­o­rie, is an astonishingly comprehensive and insightful survey for beginning students of the present state of syntactic theory. Wolfgang Sternefeld und Frank Richter, Zeitschrift für Sprach­wissen­schaft, 2012
%
% This is the kind of work that has been sought after for a while [...] The impartial and objective discussion offered by the author is particularly refreshing. Werner Abraham, Germanistik, 2012
}