\btitle{A grammar of Yakkha}
\bsubtitle{}
\bauthor{Diana Schackow}
\bisbnsc{978-3-946234-13-5
}
\bisbnhc{978-3-946234-12-8
} 
\bprice{35}{70}
\bpages{622
}
\bappeared{2015}
\blanguage{English}
\btagline{This grammar provides the first comprehensive grammatical description of Yakkha, a Sino-Tibetan language of the Kiranti branch. }
\bblurb{ This grammar provides the first comprehensive grammatical description of Yakkha, a Sino-Tibetan language of the Kiranti branch.
Yakkha is spoken by about 14,000 speakers in eastern Nepal, in the Sankhuwa Sabha and Dhankuta districts. The grammar is based on original fieldwork in the Yakkha community. Its primary source of data is a corpus of 13,000 clauses from narratives and naturally-occurring social interaction which the author recorded and transcribed between 2009 and 2012. Corpus analyses were complemented by targeted elicitation. The grammar is written in a functional-typological framework. It focusses on morphosyntactic and semantic issues, as these present highly complex and comparatively under-researched fields in Kiranti languages.   }